\documentclass[9pt]{article}
\usepackage{fullpage}
\usepackage{amsmath}
\usepackage{amssymb}
\usepackage[usenames]{color}
\usepackage{ragged2e}
\usepackage{hyperref}


\leftmargin=0.25in
\oddsidemargin=0.25in
\textwidth=6.0in
\topmargin=-0.25in
\textheight=9.25in

\raggedright

 %m\pagenumbering{arabic}
\pagestyle{empty}

\def\bull{\vrule height 0.8ex width .7ex depth -.1ex }
% DEFINITIONS FOR RESUME

\newenvironment{changemargin}[2]{%
  \begin{list}{}{%
    \setlength{\topsep}{0pt}%
    \setlength{\leftmargin}{#1}%
    \setlength{\rightmargin}{#2}%
    \setlength{\listparindent}{\parindent}%
    \setlength{\itemindent}{\parindent}%
    \setlength{\parsep}{\parskip}%
  }%
  \item[]}{\end{list}
}

\newcommand{\lineover}{
  \begin{changemargin}{-0.05in}{-0.05in}
    \vspace*{-8pt}
    \hrulefill \\
    \vspace*{-2pt}
  \end{changemargin}
}

\newcommand{\header}[1]{
  \begin{changemargin}{-0.5in}{-0.5in}
    \scshape{#1}
    \lineover
  \end{changemargin}
}

\newcommand{\contact}[4]{
  \begin{changemargin}{-0.5in}{-0.5in}
    \begin{center}
      {\Large \scshape {#1}} \\ \smallskip
      {#2} \\ \smallskip
      {#3} \\ \smallskip
      {#4} \\ \smallskip
%      {#5} \\ \smallskip
    \end{center}
  \end{changemargin}
}

\newenvironment{body} {
  \vspace*{-16pt}
\begin{changemargin}{-0.25in}{-0.5in}
  }
{\end{changemargin}
}


\newcommand{\school}[4]{
  \textbf{#1} \hfill \emph{#2\\}
  #3\\
  #4\\
}

\newenvironment{BodyText} {
  \raggedleft
  \centering
  %\begin{it}{}
}
{%\end{it}
}


% END RESUME DEFINITIONS

\begin{document}

%%%%%%%%%%%%%%%%%%%%%%%%%%%%%%%%%%%%%%%%%%%%%%%%%%%%%%%%%%%%%%%%%%%%%%%%%%%%%%%%
% Name
\contact{George Lewis}
        {220 East 18th Street, New York, NY}
        {ghl227@gmail.com}
        {(203) 918-4859}
        %{\url{https://spontaneoussymmetry.com}}
\smallskip

\begin{body}
  \vspace{14pt}
  \textbf{Summary:}{} Data scientist and software engineer experienced in designing and deploying statitical models, building efficient data pipelines for analytics, and creating tools to improve developer productivity.

\end{body}

\smallskip

%%%%%%%%%%%%%%%%%%%%%%%%%%%%%%%%%%%%%%%%%%%%%%%%%%%%%%%%%%%%%%%%%%%%%%%%%%%%%%%%
% Experience
\header{Experience}

\begin{body}

  \vspace{14pt}

  \textbf{LendUp} \hfill 2013 - Present \\
  Head of Risk and Analytics \\
  Principal Data Scientist
  \begin{itemize}

  %% 2013-03-01 00:00:00 |    621 |   149960
  %% 2015-10-01 00:00:00 | 105142 | 28642760
  %% 2017-11-01 00:00:00 | 112478 | 37921910

    \item Founded the Risk and Analytics Team, later the Data Science Team.  Designed and implemented the risk and underwriting program, growing company's loan portfolio by a factor of 200 while decreasing loss rates and achieving unit profitability.  Led the team from seed round funding through series B. \\

    \item Designed, trained, deployed, and monitored statistical models using logistic regression, generalized additive models, decision trees, and random forests to evaluate the credit risk of all applicants and to identify and mitigate fraud. \\

    \item Built and leveraged analytics capabilities to ensure model accuracy, to track business metrics, to reveal trends in our loan portfolio, and to uncover opporitinities for our risk program. % uncover customer insights. % Continuously analyzed risk program and loan portfolio to ensure model accuracy, to track business metrics, and uncover  \\

    \item Grew the Data Science Team to more than 10 members. Established best practices for building statistical models and performing analytics.  Enforced high software standards and helped develop a culture of strong programming practices within the Data Science Team. \\

    \item Built production-quality software in Java, Python, and Clojure to implement our risk program, serve and score models statistical models, and to ingest and process data in real time. \\

    \item Recruited and trained the original Data Engineering Team.  Collaborated with Data Engineering Team to build an ETL framework, jobs, and pipelines for supporting machine learning and analytics using Python, Scala, Airflow and Spark.  Developed tooling to enable streamlined model deployment for data scientists. \\
  \end{itemize}

        \medskip

        \textbf{ATLAS Experiment, CERN} \hfill 2008 - 2013\\
        Graduate researcher on the Large Hadron Collider (LHC) \\
        %Worked on the Large Hadron Collider (LHC) at the European Organization for Nuclear Research (CERN) \\
        \medskip

        \begin{itemize}

          %% ttbar analysis
        \item Built sophisticated models using custom likelihood functions and modern frequentist inference techniques to produce world accurate measurements of fundamental particle properties.

          %% HistFactory + RooStats
        \item Developed and maintained a statistical framework in C++ that implemented novel modeling techniques and was used extensively throughout a 3,000 person collaboration.

         %% PROOF + SFRAME + Grid
        \item Used parallel batch computing to analyze petabytes of data distributed worldwide across data centers.

         %%
        \item Wrote production C++ code to select interesting collisions of the 10 million created per second.

         %% Code clean-up
        \item Enforced coding standards, conventions, and best practices across a large C++ and Python code base.

        \item Recipient of National Science Foundation US LHC Graduate Student Support Award.

        \end{itemize}

\end{body}

\smallskip

%%%%%%%%%%%%%%%%%%%%%%%%%%%%%%%%%%%%%%%%%%%%%%%%%%%%%%%%%%%%%%%%%%%%%%%%%%%%%%%%
\header{Education}

\begin{body}
  \vspace{14pt}
  \textbf{Ph.D. in Experimental High Energy Particle Physics }{} \hfill 2013{} \\
  New York University \\
  \medskip
  \textbf{B.A. in Physics and Mathematics} \hfill 2007 \\
  Columbia University\\
\end{body}

%%%%%%%%%%%%%%%%%%%%%%%%%%%%%%%%%%%%%%%%%%%%%%%%%%%%%%%%%%%%%%%%%%%%%%%%%%%%%%%%
% Skills
\header{Skills}
\begin{body}
  \vspace{14pt}
  \textbf{Modeling:}{} Linear models, decision trees, ensembles, boosting, neural networks, MCMC, hypothesis testing and confidence intervals, custom likelihoods and Bayesian modeling. \\
  \textbf{Languages:}{} Python, Java, SQL, Scala, Clojure, C++.  Some experience with R, C, Cython. \\
  %\smallskip
  \textbf{Tools:}{} Scikit-Learn, Pandas, Scipy, Jupyter, Postgres, Redshift, BigQuery, Spark, Airflow, AWS, Weka, XGBoost, Matplotlib, Seaborn, Flask, Play, Ring, Tensorflow, Tidyverse, PyMC3. \\
\end{body}

\smallskip


%\header{Portfolio}
\header{}
\begin{body}
  \vspace{14pt}
  For a sample of my work and code, please see: {\url{https://spontaneoussymmetry.com/work}}
\end{body}

\end{document}
