\documentclass[9pt]{article}
\usepackage{fullpage}
\usepackage{amsmath}
\usepackage{amssymb}
\usepackage[usenames]{color}

\leftmargin=0.25in
\oddsidemargin=0.25in
\textwidth=6.0in
\topmargin=-0.25in
\textheight=9.25in

\raggedright

\pagenumbering{arabic}

\def\bull{\vrule height 0.8ex width .7ex depth -.1ex }
% DEFINITIONS FOR RESUME

\newenvironment{changemargin}[2]{%
  \begin{list}{}{%
    \setlength{\topsep}{0pt}%
    \setlength{\leftmargin}{#1}%
    \setlength{\rightmargin}{#2}%
    \setlength{\listparindent}{\parindent}%
    \setlength{\itemindent}{\parindent}%
    \setlength{\parsep}{\parskip}%
  }%
  \item[]}{\end{list}
}

\newcommand{\lineover}{
	\begin{changemargin}{-0.05in}{-0.05in}
		\vspace*{-8pt}
		\hrulefill \\
		\vspace*{-2pt}
	\end{changemargin}
}

\newcommand{\header}[1]{
	\begin{changemargin}{-0.5in}{-0.5in}
		\scshape{#1}\\
  	\lineover
	\end{changemargin}
}

\newcommand{\contact}[4]{
	\begin{changemargin}{-0.5in}{-0.5in}
		\begin{center}
			{\Large \scshape {#1}}\\ \smallskip
			{#2}\\ \smallskip 
			{#3}\\ \smallskip
			{#4}\smallskip
		\end{center}
	\end{changemargin}
}

\newenvironment{body} {
	\vspace*{-16pt}
	\begin{changemargin}{-0.25in}{-0.5in}
  }	
	{\end{changemargin}
}	


\newcommand{\school}[4]{
	\textbf{#1} \hfill \emph{#2\\}
	#3\\ 
	#4\\
}

\newenvironment{BodyText} {
  \raggedleft
  \centering
  %\begin{it}{}
}
{%\end{it} 
}


% END RESUME DEFINITIONS

\begin{document}

%%%%%%%%%%%%%%%%%%%%%%%%%%%%%%%%%%%%%%%%%%%%%%%%%%%%%%%%%%%%%%%%%%%%%%%%%%%%%%%%
% Name
\contact{George H. Lewis}{ghl227@nyu.edu}{(203) 918-4859}


%%%%%%%%%%%%%%%%%%%%%%%%%%%%%%%%%%%%%%%%%%%%%%%%%%%%%%%%%%%%%%%%%%%%%%%%%%%%%%%%
% Objective
\header{Objective}

\begin{body}
	\vspace{14pt}
	An exciting job requiring a unique computational, analytic, statistical, and technologic skill set.
\end{body}

\smallskip


%%%%%%%%%%%%%%%%%%%%%%%%%%%%%%%%%%%%%%%%%%%%%%%%%%%%%%%%%%%%%%%%%%%%%%%%%%%%%%%%
% Education
\header{Education}

\begin{body}
	\vspace{14pt}
	\textbf{Expected PhD in Experimental High Energy Particle Physics }{} \hfill \emph{Spring 2013}{} \\
	\emph{New York University}, New York, NY{} \\
  \medskip
	\textbf{B.A. Physics and Mathematics} \hfill \emph{Spring 2007} \\
	\emph{Columbia University}, New York, NY\\
\end{body}

\smallskip


%%%%%%%%%%%%%%%%%%%%%%%%%%%%%%%%%%%%%%%%%%%%%%%%%%%%%%%%%%%%%%%%%%%%%%%%%%%%%%%%
% Experience
\header{Experience}

\begin{body}
	\vspace{14pt}
	\textbf{Graduate Research} \hfill \emph{Fall 2007 - Present}\\
	New York University \\
        The European Organization for Nuclear Research (CERN) \\

	\vspace*{-4pt}

	\begin{itemize} \itemsep -0pt  % reduce space between items
		\item Active member of the ATLAS collaboration.
                \item Contributor to the ROOT/RooStats Project
	\end{itemize}

        \begin{quotation}
          \raggedright
          ATLAS is one of the multi-purpose experiments built to detect, analyze, and discover the products of 7-8 TeV proton-proton collisions at the Large Hadron Collider (LHC) at the European Organization for Nuclear Research (CERN) near Geneva, Switzerland.
          The High Energy Physics group at New York University has played a leading role in data aquisition, processing, analysis, and statistical modeling.
          The primary focus of our group since the startup on the LHC program has been precise measurements of the Top Quark and the discovery of the Higgs Boson.
          In particular, we have been extremely successful at developing statistical tools and techniques that have become the primary means of making precise measurements, claiming discovery, and setting limits on proposed physical models. \\
          \par
          As an active member of both the ATLAS collaboration and the High Energy Physics group at NYU, I have worked in a wide variety of roles during my graduate carreer.
          Between taking graduate courses at NYU in 2008 and 2009, I pioneered a data-driven technique for measuring a primary source of systematic uncertainty in Higgs boson searches.
          In 2010, I became the primary developer for the Missing Transverse Energy trigger, which uses calorimeter energy to select interesting collission events that should be recorded to disk instead of being discarded.
          Algorithms used to trigger the saving of events must be extremely fast, since collission occur at the LHC as often as every 25 ns, and must be reliable and robust to ensure a high data-taking efficiency.
          In addition, I developed, wrote, and tested a new class of Missing Energy triggers, based on Missing Energy Significance, that provided a much higher signal-to-noise rejection for many important physical processes.
          These triggers were approved of by ATLAS management and have been a component of data aquisition ever since.
          

        \end{quotation}
        
        \medskip

	\textbf {Teaching Assistant} \hfill \emph{Fall 2007 - Fall 2008}\\
        New York University \\
	\vspace*{-4pt}
	\begin{itemize} \itemsep -0pt
		\item Graduate Quantum Mechanics I
		%\item Quarks to the Cosmos
                \item General Physics I
	\end{itemize}

        \medskip

	\textbf {Undergraduate Research} \hfill \emph{Summer 2006}\\
        Los Alamos National Laboratory \\
        Stanford Linear Accelerator (SLAC) \\

\end{body}

\smallskip

% \newpage{} % uncomment this line if you want to force a new page


%%%%%%%%%%%%%%%%%%%%%%%%%%%%%%%%%%%%%%%%%%%%%%%%%%%%%%%%%%%%%%%%%%%%%%%%%%%%%%%%
% Publications
\header{Selected Publications}

\begin{body}
  \vspace{14pt}
  % http://inspirehep.net/record/1090422?ln=en
  The ATLAS Collaboration. ``\textbf{Search for same-sign top-quark production and fourth-generation down-type quarks in pp collisions at $\sqrt{s}=7$ TeV with the ATLAS detector},'' in \emph{Journal of High Energy Physics (JHEP) 1204 (2012) 069}. 2012.\\
  \medskip
  % https://atlas.web.cern.ch/Atlas/GROUPS/PHYSICS/CONFNOTES/ATLAS-CONF-2012-130/
  The ATLAS Collaboration. ``\textbf{Search for exotic same-sign dilepton signatures ($b'$ quark, $T5/3$ and four top quarks production) in 4.7 $fb^{-1}$ of pp collisions at $\sqrt{s}=7$ TeV with the ATLAS detector},'' in \emph{ATLAS-CONF-2012-130}. 2012.\\
  \medskip
  % https://atlas.web.cern.ch/Atlas/GROUPS/PHYSICS/CONFNOTES/ATLAS-CONF-2012-024/
  The ATLAS Collaboration.  ``\textbf{Statistical combination of top quark pair production cross-section measurements using dilepton, single-lepton, and all-hadronic final states at $\sqrt{s}=7$ TeV with the ATLAS detector},'' in \emph{ATLAS-CONF-2012-024}. 2012. \\
  \medskip
  % https://cdsweb.cern.ch/record/1456844?ln=en
  Kyle Cranmer, George Lewis, Lorenzo Moneta, Akira Shibata, Wouter Verkerke.  ``\textbf{HistFactory: A tool for creating statistical models for use with RooFit and RooStats}'' in \emph{CERN-OPEN-2012-016}. 2012. \\
  \medskip
  % https://atlas.web.cern.ch/Atlas/GROUPS/PHYSICS/CONFNOTES/ATLAS-CONF-2011-108/
  The ATLAS Collaboration.  ``\textbf{Measurement of the top quark pair production cross-section based on a statistical combination of measurements of dilepton and single-lepton final states at $\sqrt{s}=7$ TeV with the ATLAS detector},'' in \emph{ATLAS-CONF-2011-108}. 2011. \\
  \medskip
  % http://inspirehep.net/record/924315
  The ATLAS Collaboration. ``\textbf{Measurement of the top quark pair production cross section in pp collisions at $\sqrt{s}=7$ TeV in dilepton final states with ATLAS},'' in \emph{Phys Lett B707 (2012) 459-477}. 2010.\\
\end{body}

\smallskip


\header{Selected Talks and Posters}
\begin{body}
	\vspace{14pt}
        % Talks and Posters
        % http://indico.cern.ch/getFile.py/access?contribId=29&sessionId=6&resId=0&materialId=slides&confId=120126
        Diego Casadei, George Lewis. ``\textbf{Missing ET signicance (XS) triggers in ATLAS}'' Level 1 Calorimeter Joint Meeting, Cambridge University, 2011. \\
        \medskip
        % https://cdsweb.cern.ch/record/1336182/files/ATL-COM-PHYS-2011-292.pdf?
        George Lewis.  ``\textbf{New ATLAS Triggers Based on the Missing ET Significance}'' Large Hadron Collider Conference (LHCC), CERN, 2011. \\
\end{body}

\smallskip



%%%%%%%%%%%%%%%%%%%%%%%%%%%%%%%%%%%%%%%%%%%%%%%%%%%%%%%%%%%%%%%%%%%%%%%%%%%%%%%%
% Skills
\header{Skills}

\begin{body}
	\vspace{14pt}
	\emph{\textbf{Programming:}}{} C++/C, Python, Numpy/Scipy, Javascript, HTML/CSS, \LaTeX, PHP, git/svn, bash \\
	\emph{\textbf{High Energy Physics:}}{} ROOT, RooStats, Athena\\
\end{body}

\smallskip


\end{document}
