\documentclass[9pt]{article}
\usepackage{fullpage}
\usepackage{amsmath}
\usepackage{amssymb}
\usepackage[usenames]{color}
\usepackage{ragged2e}

\leftmargin=0.25in
\oddsidemargin=0.25in
\textwidth=6.0in
\topmargin=-0.25in
\textheight=9.25in

\raggedright

\pagenumbering{arabic}

\def\bull{\vrule height 0.8ex width .7ex depth -.1ex }
% DEFINITIONS FOR RESUME

\newenvironment{changemargin}[2]{%
  \begin{list}{}{%
    \setlength{\topsep}{0pt}%
    \setlength{\leftmargin}{#1}%
    \setlength{\rightmargin}{#2}%
    \setlength{\listparindent}{\parindent}%
    \setlength{\itemindent}{\parindent}%
    \setlength{\parsep}{\parskip}%
  }%
  \item[]}{\end{list}
}

\newcommand{\lineover}{
	\begin{changemargin}{-0.05in}{-0.05in}
		\vspace*{-8pt}
		\hrulefill \\
		\vspace*{-2pt}
	\end{changemargin}
}

\newcommand{\header}[1]{
	\begin{changemargin}{-0.5in}{-0.5in}
		\scshape{#1}\\
  	\lineover
	\end{changemargin}
}

\newcommand{\contact}[4]{
	\begin{changemargin}{-0.5in}{-0.5in}
		\begin{center}
			{\Large \scshape {#1}}\\ \smallskip
			{#2}\\ \smallskip 
			{#3}\\ \smallskip
			{#4}\smallskip
		\end{center}
	\end{changemargin}
}

\newenvironment{body} {
	\vspace*{-16pt}
	\begin{changemargin}{-0.25in}{-0.5in}
  }	
	{\end{changemargin}
}	


\newcommand{\school}[4]{
	\textbf{#1} \hfill \emph{#2\\}
	#3\\ 
	#4\\
}

\newenvironment{BodyText} {
  \raggedleft
  \centering
  %\begin{it}{}
}
{%\end{it} 
}


% END RESUME DEFINITIONS

\begin{document}

%%%%%%%%%%%%%%%%%%%%%%%%%%%%%%%%%%%%%%%%%%%%%%%%%%%%%%%%%%%%%%%%%%%%%%%%%%%%%%%%
% Name
\contact{George H. Lewis}{ghl227@nyu.edu}{(203) 918-4859}


%%%%%%%%%%%%%%%%%%%%%%%%%%%%%%%%%%%%%%%%%%%%%%%%%%%%%%%%%%%%%%%%%%%%%%%%%%%%%%%%
% Objective
\header{Objective}
\begin{body}
	\vspace{14pt}
	An exciting job requiring a unique analytic, computational, statistical, and technologic skill set.
\end{body}

\smallskip


%%%%%%%%%%%%%%%%%%%%%%%%%%%%%%%%%%%%%%%%%%%%%%%%%%%%%%%%%%%%%%%%%%%%%%%%%%%%%%%%
% Education
\header{Education}

\begin{body}
	\vspace{14pt}
	\textbf{Expected PhD in Experimental High Energy Particle Physics }{} \hfill \emph{Spring 2013}{} \\
	\emph{New York University}, New York, NY{} \\
  \medskip
	\textbf{B.A. in Physics and Mathematics} \hfill \emph{Spring 2007} \\
	\emph{Columbia University}, New York, NY\\
\end{body}

\smallskip


%%%%%%%%%%%%%%%%%%%%%%%%%%%%%%%%%%%%%%%%%%%%%%%%%%%%%%%%%%%%%%%%%%%%%%%%%%%%%%%%
% Experience
\header{Experience}

\begin{body}
	\vspace{14pt}
	\textbf{Graduate Research} \hfill \emph{Fall 2007 - Present}\\
	New York University \\
        The European Organization for Nuclear Research (CERN) \\

	\vspace*{-4pt}

%	\begin{itemize} \itemsep -0pt  % reduce space between items
%		\item Active member of the ATLAS collaboration.
%                \item Contributor to the ROOT/RooStats Project
%	\end{itemize}

        \begin{quotation}
          \justifying
          % \raggedright
          % \raggedright
          %\centering
          %\parindent=1.5em % <- or whatever indent you want

          ATLAS is one of the multi-purpose experiments built to analyze 7-8 TeV proton-proton collisions at the Large Hadron Collider (LHC), 
          which is located beneath the European Organization for Nuclear Research (CERN) near Geneva, Switzerland. 
          The High Energy Physics group at New York University has played a leading role within the ATLAS experiment in terms of data acquisition, physical analysis and statistical modeling. 
          Since the startup of the LHC, our group has focused on precision measurements of the top quark and the discovery of the Higgs boson.
          %The focus of our group’s efforts has been making precise measurements of the top quark and the discovery of the Higgs boson. 
          In particular, we have been extremely successful at developing statistical tools and techniques that have become the primary means of making precise measurements, 
          claiming discovery, and setting limits on proposed physical models.

          \smallskip

          As an active member of both the ATLAS collaboration and the High Energy Physics group at NYU, I have worked in a wide variety of roles during my graduate career. 
          Between taking graduate courses at NYU in 2008 and 2009, I pioneered a data-driven technique for measuring a major source of systematic uncertainty in Higgs boson searches. 
          In 2010, I became the primary code developer for the Missing Transverse Energy trigger, 
          which uses calorimeter energy to select interesting collision events to be recorded and analyzed.  
          Algorithms designed to trigger the saving of events at ATLAS must be extremely fast, since collision occur at the LHC as often as every 25 ns, 
          and they must be reliable and robust to ensure a high data-taking efficiency. 
          In this role, I developed, wrote, and tested a new class of triggers, based on a variable called Missing Energy Significance, 
          that provided a much stronger background rejection for many important physical signals. 
          Since their development, these triggers have been running as an important component of online data taking.

          \smallskip

          Since 2010, my analysis has focused on performing extremely precise measurements and searching for new physics using collision events that produce top quarks.  
          Events involving the top quark are a fantastic environment for understanding the performance ATLAS detector, 
          making accurate measurements of the Standard Model, and searching for or constraining new physical models. 
          Much of my graduate work has been related to measuring the cross-section of top quark pair production. 
          In particular, I took on a leadership role in performing statistical combinations of individual cross-section measurements 
          which lead to the experiment's most accurate measurement of the top quark cross-section.
          %to provide accurate measurements that incorporated multiple physical channels. 
          In addition, I have performed searches for exotic physical models using final states that contain two top quarks of the same electric change. 
          These events have little background from the Standard Model, which makes them an ideal means of searching for new physical phenomena.
          My most important role in this analysis was quantifying the size of statistical and systematic uncertainties and properly incorporating those uncertainties into measurements.

          \smallskip

          As a common thread across my various roles in the ATLAS collaboration, I have focused on developing statistical tools and techniques that have become widely used within the collaboration. 
          I am currently the primary developer of a statistical modeling package called “HistFactory” which facilitates the building of large probabilistic models 
          and carefully incorporating systematic uncertainties. 
          It is used by many analyses within ATLAS (as well as CMS, our sister experiment), including the analysis that lead to the discovery of the Higgs boson.  
          Being a local expert on statistics has allowed me to advise or participate in a wide variety of analyses across the experiment.

        \end{quotation}
        
        \medskip

	\textbf {Teaching Assistant} \hfill \emph{Fall 2007 - Fall 2008}\\
        New York University \\
	\vspace*{-4pt}
	\begin{itemize} \itemsep -0pt
		\item Graduate Quantum Mechanics I
		\item Quarks to the Cosmos
                \item General Physics I
	\end{itemize}

        \medskip

	\textbf {Undergraduate Research} \hfill \emph{Summer 2006}\\
        Los Alamos National Laboratory \\
        Stanford Linear Accelerator (SLAC) \\

        \begin{quotation}
          \justifying
          
          As an undergradute, I worked with a group of physicists and chemists at Los Alamos National Laboratory in New Mexico 
          to study the electron orbital structure of heavy metallic elemenets, in particular Uranium.
          To probe the structure of orbital shells, we used a technique called XANES which uses high energy x-rays to expel the innner most electrons in uranium atoms.
          The rate of absorption of these x-rays provides a wealth of information about the atom's electronic structure.
          I played a central role within our small group in the design and construction of our table-top experiment at Los Alamos.
          Once constructed, we trasported our experiment to the Stanford Linear Accelerator, where we used the beam's syncatron radiation as the source of our x-rays.

        \end{quotation}

\end{body}

\smallskip

% \newpage{} % uncomment this line if you want to force a new page

%\clearpage
%\newpage

%%%%%%%%%%%%%%%%%%%%%%%%%%%%%%%%%%%%%%%%%%%%%%%%%%%%%%%%%%%%%%%%%%%%%%%%%%%%%%%%
% Publications
\header{Selected Publications}

\begin{body}
  \vspace{14pt}
  \begin{itemize}
  % https://atlas.web.cern.ch/Atlas/GROUPS/PHYSICS/CONFNOTES/ATLAS-CONF-2012-130/
  \item The ATLAS Collaboration. \textbf{Search for exotic same-sign dilepton signatures ($b'$ quark, $T5/3$ and four top quarks production) in 4.7 $fb^{-1}$ of pp collisions at $\sqrt{s}=7$ TeV with the ATLAS detector} \\
    ATLAS-CONF-2012-130  \hfill \emph{Summer 2012} \\
    \medskip
    % https://cdsweb.cern.ch/record/1456844?ln=en
  \item Kyle Cranmer, George Lewis, Lorenzo Moneta, Akira Shibata, Wouter Verkerke. \textbf{HistFactory: A tool for creating statistical models for use with RooFit and RooStats} \\
    CERN-OPEN-2012-016 \hfill \emph{Spring 2012} \\
    \medskip
    % https://atlas.web.cern.ch/Atlas/GROUPS/PHYSICS/CONFNOTES/ATLAS-CONF-2012-024/
  \item The ATLAS Collaboration.  \textbf{Statistical combination of top quark pair production cross-section measurements using dilepton, single-lepton, and all-hadronic final states at $\sqrt{s}=7$ TeV with the ATLAS detector} \\
    ATLAS-CONF-2012-024 \hfill \emph{Spring 2012} \\
    \medskip
    % http://inspirehep.net/record/1090422?ln=en
  \item The ATLAS Collaboration. \textbf{Search for same-sign top-quark production and fourth-generation down-type quarks in pp collisions at $\sqrt{s}=7$ TeV with the ATLAS detector} \\ 
    JHEP 1204 (2012) 069 \hfill \emph{Winter 2011} \\
    \medskip
    % https://atlas.web.cern.ch/Atlas/GROUPS/PHYSICS/CONFNOTES/ATLAS-CONF-2011-108/
  \item The ATLAS Collaboration.  \textbf{Measurement of the top quark pair production cross-section based on a statistical combination of measurements of dilepton and single-lepton final states at $\sqrt{s}=7$ TeV with the ATLAS detector} \\
    ATLAS-CONF-2011-108 \hfill \emph{Summer 2011} \\
    \medskip
    %http://inspirehep.net/record/880002
    \item The ATLAS Collaboration. \textbf{Measurement of the top quark-pair production cross section with ATLAS in pp collisions at $\sqrt{s}=7$ TeV} \\
      EPJC 71 (2011) 1577 \hfill \emph{Fall 2010} \\

%    \medskip
%    % http://inspirehep.net/record/924315
%  \item The ATLAS Collaboration. \textbf{Measurement of the top quark pair production cross section in pp collisions at $\sqrt{s}=7$ TeV in dilepton final states with ATLAS} \\
%    Phys Lett B707 (2012) 459-477 \hfill  \\

  \end{itemize}
\end{body}

\smallskip

\header{Selected Talks and Posters}
\begin{body}
	\vspace{14pt}
        % Talks and Posters
        % http://indico.cern.ch/getFile.py/access?contribId=29&sessionId=6&resId=0&materialId=slides&confId=120126
        \textbf{Missing ET signicance (XS) triggers in ATLAS} \\
        Level 1 Calorimeter Joint Meeting, Cambridge University \hfill \emph{Spring 2011} \\
        \medskip
        % https://cdsweb.cern.ch/record/1336182/files/ATL-COM-PHYS-2011-292.pdf?
        \textbf{New ATLAS Triggers Based on the Missing ET Significance} \\
        Large Hadron Collider Conference (LHCC), CERN \hfill \emph{Spring 2011} \\
        \medskip
        % https://indico.cern.ch/conferenceDisplay.py?confId=39694
        % https://indico.cern.ch/getFile.py/access?contribId=11&resId=0&materialId=slides&confId=39694
        \textbf{Measuring Central Jets in EW and QCD Z+jets} \\
        Standard Model Plenary Session, CERN \hfill  \emph{Fall 2008} \\
        \medskip
        % https://indico.cern.ch/conferenceDisplay.py?confId=35673
        % https://indico.cern.ch/getFile.py/access?contribId=6&resId=0&materialId=slides&confId=35673
        \textbf{Extracting Central Jet Veto Efficiency with Data} \\
        Higgs Working Group Meeting, CERN \hfill \emph{Summer 2008} \\

\end{body}

\smallskip


\header{Awards}
\begin{body}
  \vspace{14pt}
  \textbf{National Science Foundation US LHC Graduate Student Support Award} \hfill \emph{2010} \\
\end{body}

%\clearpage
%\newpage
%%%%%%%%%%%%%%%%%%%%%%%%%%%%%%%%%%%%%%%%%%%%%%%%%%%%%%%%%%%%%%%%%%%%%%%%%%%%%%%%
% Skills
\header{Skills}

\begin{body}
	\vspace{14pt}
        \emph{\textbf{General:}}{} Analysis of large datasets, statistical modeling, data visualization, web design, collaborating with collegues around the world, object-oriented design, graduate level physics and advanced mathematics \\
        \smallskip
	\emph{\textbf{Programming:}}{} C++/C, Python, Numpy/Scipy, Javascript, HTML/CSS, \LaTeX, PHP, git/svn, bash \\
        \smallskip
	\emph{\textbf{High Energy Physics Tools:}}{} ROOT, RooFit, RooStats, HistFactory, Athena, PAthena and the grid \\
\end{body}

\smallskip


\end{document}
